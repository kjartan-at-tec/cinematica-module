\documentclass[border=10pt,varwidth]{standalone}
\usepackage{tikz}
\usetikzlibrary{patterns}

  \newcommand*{\link}[4]{%
  \draw[thick, color=#3, rounded corners] (-3pt, -#2/2) rectangle (#1+3pt, #2/2);
  \draw (0,0) circle[radius=1pt];
  \draw (#1, 0) circle[radius=1pt];
  \node at (#1/2, 1.3*#2) {#4};}

\newcommand*{\slidelink}[4]{%
  \draw[thick, color=#3, rounded corners] (-4pt, -#2/2) rectangle (#1+3pt, #2/2);
  \draw[] (-2pt, -1pt) rectangle (#1, 1pt) ;
  \node at (#1/2, 1.3*#2) {#4};}

\newcommand*{\tool}[2]{%
  \draw[ultra thick, color=#2] (0,0) to (#1, 0);}

\newcommand*{\refframe}[4]{%
  \draw[->, thick, #4] (0,0) to (#1, 0) node[right]{#2};
  \draw[->, thick, #4] (0,0) to (0, #1) node[above]{#3};} 

\newcommand*{\robotbase}[2]{%
    \draw[thick, black!40, rounded corners=8pt] (-#1,-#2)-- (-#1, 0) --
        (0,#2)--(#1,0)--(#1,-#2);
    \draw (-0.3,-#2)-- (0.3,-#2);
    \fill[pattern=north east lines] (-0.3,-0.3) rectangle (0.3,-#2);
  }

\begin{document}

\pgfmathsetmacro{\qone}{0.9}
\pgfmathsetmacro{\qtwo}{30}
\pgfmathsetmacro{\Lslide}{1.6}
\pgfmathsetmacro{\Lone}{2}
\pgfmathsetmacro{\Ltwo}{1}
\pgfmathsetmacro{\Ltwostart}{\Lone}
\pgfmathsetmacro{\lwidth}{0.22}
\pgfmathsetmacro{\lwidthtwo}{0.16}

\begin{tikzpicture}[scale=2,]

  \robotbase{0.2cm}{0.2cm}

  \draw[->] (-1, 0) to (5,0) node[right] {$x$};
  \draw[->] (0,-1) to (0,3) node[above] {$y$};

  \begin{scope}[rotate=\qtwo]
      
  \slidelink{\Lslide cm}{\lwidth}{black!80}{}

  \foreach \xx/\clr in {0/red!50!white, \qone/red!70!black} {
  %\foreach \xx/\clr in {0/red!70!black} {
  \begin{scope}[xshift=\xx cm]
    \link{\Lone cm}{\lwidthtwo}{\clr}{}
    %\draw[dashed] (\Lone/2, 0) to (1.5*\Lone, 0);

    \draw[dashed] (0, 0) to (0, -0.6);

    \begin{scope}[xshift=\Lone cm]
        %\tool{0.3 cm}{red!black!80}
        \refframe{0.5 cm}{$x_e$}{$y_e$}{\clr}
      \end{scope}

      \begin{scope}[rotate=-\qtwo]
      \refframe{0.5 cm}{$x_l$}{$y_l$}{\clr}
    \end{scope}
  \end{scope}
  }

    
  \draw[<->] (0, -0.5) -- node[below] {$\theta_1$} ++(\qone, 0);


    \end{scope}

  \end{tikzpicture}    
\end{document}
